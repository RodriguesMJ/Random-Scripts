\chapter*{Abstract}
\addcontentsline{toc}{chapter}{Abstract}
The catalytic mechanism of the Pdx1 subunit of the pyridoxal 5'-phosphate (PLP\glsadd{PLP}) synthase enzyme complex has been the subject of intense study since its discovery in 1999. Pdx1 uses two active sites (P1 and P2) to perform a complex reaction combining ribose 5-phosphate\glsadd{R5P}, ammonia and glyceraldehyde 3-phosphate\glsadd{G3P} to form PLP. 

While some aspects of the Pdx1 mechanism are now understood, several questions remain, in particular how the enzyme transfers the reaction from one active site to the next and how the reactions in the two sites are co-ordinated. 

In this investigation, X-ray crystallography and UV-Vis spectrophotometry have been used to determine the structure of the protein in various intermediate states and elucidate the catalytic mechanism. The role of specific active site residues in catalysis of PLP biosynthesis by the \textit{Arabidopsis thaliana} Pdx1 protein is investigated using site-directed mutagenesis.   

Chapter \ref{ch:Intro} provides an introduction to the biological role of pyridoxal 5'-phosphate and its biosynthesis by the PLP synthase enzyme complex. Chapter \ref{ch:Intro} also describes the use of X-ray crystallography to understand how proteins function and how complementary methods such as UV-Vis\glsadd{UV-Vis} spectrophotometry can be used to track the effects of site-specific radiation damage during collection of X-ray diffraction data. The methods used to produce, purify and investigate the Pdx1 enzyme are described in Chapter \ref{ch:Methods}. Chapter \ref{ch:WT_Results} provides an analysis of the activity of wild type \atpdx~ using UV-Vis spectrophotometry and X-ray crystallography to monitor the accumulation of chromophoric intermediates and the product, PLP. Chapter \ref{ch:Mutant_Results} describes the use of site-directed mutagenesis to trap the Pdx1 protein in additional intermediate states and the characterisation of these intermediate states using UV-Vis spectroscopy and crystallography.

UV-Vis spectra of Pdx1 crystals were often collected to ensure that the Pdx1 enzyme was in the desired intermediate state, before collection of X-ray diffraction data. It became clear that the UV-Vis spectra of the Pdx1 crystals in some intermediate states were changing during X-ray data collection. Chapter \ref{ch:multi_xtal} describes the series of experiments that were performed to check that the Pdx1 structures obtained were not affected by site-specific radiation damage. The use of multi-crystal data collection strategies and UV-Vis spectroscopy showed that changes in the spectra were instead caused by radiolysis of the solvent surrounding the crystals.\clearpage\null\newpage
%Some of the Pdx1 intermediate states are chromophoric and have been characterised using combinations of X-ray crystallography and UV-Vis spectroscopy.  

%Addition of the first two substrates to Pdx1 leads to accumulation of an intermediate termed I320     