\chapter{Conclusions}\label{ch:Conclusions}

The primary aim of this project was to improve our understanding of the reaction mechanism of the Pdx1 enzyme by determining the structure of the protein as the enzyme progresses through the catalytic cycle. Combining the crystal structures of \textit{At}Pdx1.3 obtained during this investigation with structures from other members of the Tews group and published data it is possible to produce an updated mechanism for Pdx1 that better explains how the enzyme transfers intermediates between the P1 and P2 sites.

\textit{At}Pdx1 is observed to bind R5P in the P1 active site with a covalent link between the $\varepsilon$ nitrogen of the Lysine 98 side chain and C1 of R5P, as has been observed in other Pdx1-R5P structures (Fig \ref{fig:Pdx1_R5P}) \cite{Guedez2012,Smith2015}. The channelling of ammonia through the hydrophobic tunnel in the center of Pdx1 causes a change in the conformation of the $\beta$6 strand. This causes Lysine 166 to re-orient away from the P2 site and towards the P1 site. This conformational change has been observed in both the wild type protein and the \atpdx ~K166R mutant. Formation of the I320 intermediate is dependent on the reaction between C5 of the carbohydrate intermediate and the side chain of Lys166; in the case of the \atpdx ~K166R mutant, this could not occur, enabling the trapping of a pre-I320 intermediate state. Despite nitrogen incorporation taking place, the phosphate group originating from R5P was not eliminated showing that the side chain of Lysine 166 is necessary for this step in the reaction. The structure of the \atpdx -I320/G3P intermediate produced by Dr Volker Windeisen demonstrates that G3P binds in the P1 site of Pdx1, leading to the release of the intermediate and transfer of the reaction to the P2 site. \glsadd{Pbpdx1}   \glsadd{Gspdx1} 

%The mutant structures have demonstrated that Aspartate 41 is necessary for nitrogen incorporation, but is not essential for the covalent binding of R5P. and that Lysine 98 is required for covalent binding of R5P, explaining why these mutants have not been observed to catalyse formation of I320. 

The use of covalent tethers by enzymes to transfer intermediates is prevalent in biosynthetic pathways that have evolved to perform multi-step reactions. Prominent examples include the polyketide synthase pathways (PKS), non-ribosomal polypeptide synthases (NRPS) and fatty acid synthase (FAS) \cite{Wu2001,Tanovic2008,Jenni2007}. Similarly to these enzymes, Pdx1 uses the covalent tethers to prevent diffusion of reaction intermediates out of the active site. The structures presented here demonstrate that the Pdx1 enzyme is also able to maintain precise control of the reaction by binding to the reaction intermediates covalently. In contrast to the pathways mentioned previously, Pdx1 binds to intermediates directly by forming imine bonds between active site lysine residues and the intermediates, rather than via prosthetic groups. The relay mechanism employed by Pdx1 transfers the intermediates between the P1 and P2 sites within a subunit, whereas the PKS, NRPS, and FAS enzymes use their longer tethers to transfer intermediates between subunits.

Despite the progress that has been made into understanding the mechanism of PLP biosynthesis, the complete story has not been uncovered. The steps between the addition of G3P to I320 and the product state are uncharacterised. In particular, the mechanism by which the I320/G3P intermediate is released is unknown. If the release is dependent on large scale conformational changes involving the C-terminus of Pdx1, then it may not be possible to characterise these steps in the reaction using X-ray crystallography. 

Due to the large size of the Pdx1 protein in the dodecameric form (molecular weight $\sim$400 KDa), solving the atomic structure of the whole protein in different intermediate states is beyond the reach of solution state NMR. The advances in single molecule cryo-electron microscopy may allow for this technique to be used to solve the structure of Pdx1; the smallest protein structure solved using cryo-EM is $\beta$-galactosidase, which has a molecular weight of 464 KDa, at 2.2 \si{\angstrom} resolution \cite{Bartesaghi2015}. However, the use of cryo-EM would remain dependent on being able to trap the enzyme in specific intermediate states. Suitable mutants for trapping the enzyme between the I320 and the product states have not been identified.  

Rather than using structural methods dependent on trapping in a particular intermediate state to investigate the I320 - PLP transition, time-resolved mass spectrometry may be used to monitor the enzyme reaction \cite{Wilson2004}. The experiment would require reconstitution of the Pdx1-I320 intermediate in solution and addition of G3P using rapid mixing; mass spectra are then collected with millisecond time resolution with the aim of observing transient, pre-steady state intermediates. Each of the steps in the transition from I320 to PLP should have a signature mass change, for example, the condensation reaction that occurs during ring closure would be expected to eliminate a molecule of water, which would be observed as a decrease in the mass of the protein by 18 Daltons. Following the appearance and disappearance of the intermediate states would provide information that could be integrated with existing data to propose a detailed chemical mechanism for the I320 - PLP transition.  


The radiation damage investigation in Chapter \ref{ch:multi_xtal} reached the conclusion that X-rays do not damage the I320 intermediate in a site specific manner. The primary problem with using UV-Vis spectrometry to monitor site specific damage in the UV region of the spectrum has been identified as the generation of chromophoric products of solvent radiolysis. It has been observed that the absorption maxima and rate of peak formation are dependent not only on dose but also the sample thickness and concentrations of particular solvents, such as glycerol. It is, therefore, not possible to calculate a model for the effects of solvent radiolysis on individual buffer components that is valid for the buffer more complex samples such as protein crystals. This is the main barrier to being able to subtract the effects of solvent radiolysis from the UV-Vis spectra of crystals. Future radiation damage investigations using spectroscopy in the UV region of the spectrum may benefit from optimising the cryobuffer to exclude glycerol, MPD and other constituents which decompose into chromophoric compounds.      

%\newpage\null\newpage




   