\chapter{An Analysis of the Structure and Activity of Wild Type \atpdx}\label{ch:WT_Results}

The aim of this chapter was to determine the crystal structure of Pdx1 at different stages in the catalytic cycle by first crystallising the protein and then soaking the crystals with substrates to reconstitute the reaction \textit{in crystallo}. Previous work in the Tews group had established that the \textit{A. thaliana} protein \atpdx ~crystallised reproducibly and was resilient to soaking with high concentrations of the Pdx1 substrates. It was first checked that the biosynthetic reaction could be reconstituted with protein in solution, before moving onto crystallographic experiments. The crystal structure of wild type \atpdx ~is presented here in complex with R5P, in the I320 and I320/G3P intermediate states and covalently bound to the product, PLP.   

\section{Spectra of wild-type \textit{At}Pdx1.3 in native, I320 and PLP bound states.}
The enzymatic activity of \atpdx ~was measured for both I320 formation and PLP biosynthesis as described in Section \ref{sec:Assay_Methods}. Spectra were collected before addition of substrates to \atpdx, after the I320 assay and after the PLP assay (Figure \ref{fig:Pdx1WT_WLspec_raw}). 

\begin{figure}[!htbp]
\centering
%\begin{minipage}{\linewidth}
	\makebox[\linewidth]{
	\includegraphics[width=8cm, height=8cm, keepaspectratio]{/Users/matt/Dropbox/ThesisPrep/Thesis/fig/spectra/wl/wt/WT_WLspec_raw.pdf}}	

	\captionof{figure}[Spectra of wild type \atpdx ~in solution]{Spectra of \atpdx ~before addition of substrates (blue), after incubation with R5P and ammonia salts (yellow), and after incubation of \atpdx -I320 with G3P (green).\label{fig:Pdx1WT_WLspec_raw}}	
%\end{minipage}
\end{figure}

The spectrum of \atpdx ~before addition of substrates shows a strong absorbance peak at 280 \nm ; this absorbance peak is observed in all proteins containing tryptophan residues. Also visible is a slight shoulder at 320 \nm ~and a small peak at ~415 \nm . The shoulder at 320 \nm ~shows that a small amount of the I320 intermediate co-purifies with the protein, while the peak ~415 \nm ~indicates that the product also co-purifies with \atpdx . These absorbance peaks suggest that neither the I320 intermediate or the product are separated from the protein during the size exclusion chromatography step of the protein purification (Section \ref{sec:SEC_Methods}), in keeping with previously published data on Pdx1 from other species \cite{Raschle2007}. It is possible to empty the enzyme active sites of I320 and PLP by incubating Pdx1 with ammonium salts and G3P for three hours followed by dialysis \cite{Raschle2007}. However, attempts to free \atpdx ~of I320 and PLP resulted in precipitation of a significant fraction of the protein, leaving too little soluble protein with which to perform the subsequent enzyme assays.   %cite Wilson 2005 for 280 peak?

After incubation with R5P and ammonia, \atpdx ~displays a more pronounced peak at 320 \nm , signalling accumulation of the \atpdx -I320 intermediate (Figure \ref{fig:Pdx1WT_WLspec_raw}). Although the 280 \nm ~peak appears to increase, this is likely to be caused by the accumulation of the I320 species and the overlap of the 280 \nm ~and 320 \nm ~peaks. After incubation of \atpdx -I320 with G3P for one hour, there is an increase in absorbance at 415 \nm ~as has been previously described \cite{Raschle2006}. 

Interpretation of \atpdx ~spectra is complicated by the presence background scattering that is often observed when protein aggregation takes place in the sample. The samples were briefly centrifuged at 13,000 RPM in a benchtop centrifuge, to reduce the effect of aggregation on the spectra; however, some aggregate remained.  

\begin{minipage}{\linewidth}
	\makebox[\linewidth]{
	\includegraphics[width=8cm, height=8cm, keepaspectratio]{/Users/matt/Dropbox/ThesisPrep/Thesis/fig/spectra/wl/wt/WT_WLspec_raysub.pdf}}	

	\captionof{figure}[Spectra of \atpdx ~after subtraction of background]{Spectra of \atpdx ~before addition of substrates (blue), after a one hour incubation with R5P and ammonia salts (yellow), and one hour after addition of G3P to \atpdx -I320 (green); after estimation and subtraction of contribution to measured absorbance from Rayleigh scattering (Methods Section \ref{sec:bg}).\label{fig:Pdx1WT_RayleighSub}}	
\end{minipage}

Figure \ref{fig:Pdx1WT_RayleighSub} shows the \atpdx ~spectra after the subtraction of the estimated contribution to the spectra from Rayleigh scattering, as described in Section \ref{sec:bg}. The increase in absorbance at 320 \nm ~as a result of the addition of R5P and ammonia to \atpdx ~becomes more apparent, as does the increase in absorbance at 415 \nm ~when G3P is added to \atpdx -I320 (Figure \ref{fig:Pdx1WT_RayleighSub}). Despite being named I320, the $\lambda _{max}$ ~of \atpdx -I320 is 312 nm in the corrected spectrum; this is similar to the \textit{B. subtilis} Pdx1-I320 \lwl of 315 nm \cite{Raschle2007}.

The background subtraction introduces a couple of artefacts. These are the negative absorbance values for the corrected spectrum of \atpdx ~before addition of substrates, between 210 nm and 240 nm, and between 300 nm and 350 nm (Figure \ref{fig:Pdx1WT_RayleighSub}). Given that it is not possible for a sample to have a negative absorbance it is clear that this is an artefact of the background subtraction. The background correction is based on two assumptions, if either assumption is not valid, the correction will produce errors. The first assumption is that the only source of absorption between 500 nm and 700 nm is Rayleigh scattering; the second is that absorbance in the region of interest is solely caused by the addition of Rayleigh scattering to absorbance by the sample. The negative absorbance in the baseline corrected spectra suggests that the contribution of light scattering to the spectra is being overestimated and that the failure of the first assumption is the source of the error. Despite these artefacts, the corrected spectra (Figure \ref{fig:Pdx1WT_RayleighSub}) are easier to interpret than the raw spectra (Figure \ref{fig:Pdx1WT_WLspec_raw}).                       

This experiment has shown that it is possible to reconstitute biosynthesis of PLP with wild type \atpdx, ~using R5P, ammonia salts and G3P as substrates, in solution.
\clearpage
\section{Wild Type \atpdx ~Activity Assays}
\label{sec:WT_Assays}
The enzymatic activity of wild type \atpdx ~was measured by addition of R5P and ammonia to the protein to form \atpdx -I320 as described in the methods (Section \ref{sec:320_assay_methods}). The change in absorbance at 315 \nm ~was measured over the course of one hour. The catalytic activity of the protein was calculated by fitting a straight line to the linear phase of the reaction, which was taken to be the first 50 seconds after the beginning of the measurement (Figure \ref{fig:pdx1_WT_ts_320_fit}). The gradient of the straight line is equal to the change in absorbance per second, knowledge of the molar extinction coefficient of I320 ($\varepsilon$ = 16,200 {\si{\per\molar\per\centi\metre}}) and protein concentration allows the calculation of the enzyme activity in units of nanomoles of I320 produced per milligram of \atpdx ~per minute \cite{Raschle2006}.  

\begin{figure}[!htbp]
\centering
\begin{subfigure}{.5\textwidth}
  \centering
  \includegraphics[width=8cm, height=8cm, keepaspectratio]{/Users/matt/Dropbox/ThesisPrep/Thesis/fig/spectra/timescan/wildtype/Pdx1_WT_320_ts_MB_triplicate.pdf}
  \caption{}
  \label{fig:pdx1_WT_ts_320}
\end{subfigure}%
\begin{subfigure}{.5\textwidth}
  \centering
  \includegraphics[width=8cm, height=8cm, keepaspectratio]{/Users/matt/Dropbox/ThesisPrep/Thesis/fig/spectra/timescan/wildtype/Pdx1_WT_PLP_ts_MB_triplicate.pdf}
  \caption{}
  \label{fig:pdx1_WT_ts_PLP}
\end{subfigure}
\begin{subfigure}{.5\textwidth}
  \centering
  \includegraphics[width=8cm, height=8cm, keepaspectratio]{/Users/matt/Dropbox/ThesisPrep/Thesis/fig/spectra/timescan/WT_Fit_50s_I320_norm.pdf}
  \caption{}
  \label{fig:pdx1_WT_ts_320_fit}
\end{subfigure}%
\begin{subfigure}{.5\textwidth}
  \centering
  \includegraphics[width=8cm, height=8cm, keepaspectratio]{/Users/matt/Dropbox/ThesisPrep/Thesis/fig/spectra/timescan/WT_Fit_200_300s_PLP.pdf}
  \caption{}
  \label{fig:pdx1_WT_ts_PLP_fit}
\end{subfigure}
\caption[\textit{In vitro} assays of wild type \atpdx~ activity]{(a) Change in absorbance at 315 nm when wild type \atpdx ~is incubated with R5P and ammonia for one hour (triplicate). (b) Change in absorbance at 415 nm when wild type \atpdx -I320 is incubated with G3P for one hour (triplicate). (c) Fit of a straight line (black) to the absorbance in the I320 assay. (d) Fit of a straight line (black) to the absorbance in the PLP assay.}
\end{figure}

The activity of wild type \atpdx~ for I320 production was 11.03 \si{\nano\mole\per\milli\gram\per\minute}~ with a standard deviation of 0.61 \si{\nano\mole\per\milli\gram\per\minute}. The rate of I320 production observed here is greater than that observed for \textit{B. subtilis} Pdx1 which has an activity of $\sim$ 1 \act \cite{Raschle2007}. It is not possible to make a direct comparison between the rate of activity \atpdx~ and \textit{Bs}Pdx1 as the previous study used a ten-fold lower concentration of R5P and used glutamine in the presence of \textit{Bs}Pdx2 as a nitrogen source \cite{Raschle2007}. The accumulation of the I320 intermediate plateaus after $\sim$400 seconds as all the active sites of all of the Pdx1 proteins in the assays contain the I320 intermediate. \textit{At}Pdx2 and glutamine were not used as an ammonia source in these experiments as recombinant expression and purification provided low yields of protein; the protein also had a low stability, both at room temperature and when frozen in glycerol. 

The assay of the PLP biosynthetic rate for wild type \atpdx~ showed a clear lag phase in the first 120 seconds of the experiment for all replicates (Figure \ref{fig:pdx1_WT_ts_PLP}). Lag phases in enzyme assays are often indicative of co-operativity playing a role in the catalytic mechanism, and the flexible C-terminus of Pdx1 has previously been assigned a role in the closing of the active site of the neighbouring subunit during conversion of I320 to PLP \cite{Raschle2009}. 

The linear phase of the reaction was taken to be between 200 and 300 seconds after the start of the measurement (Figure \ref{fig:pdx1_WT_ts_PLP_fit}). A straight line was fitted to each set of absorbance values between these time points; the gradient of the line was used to calculate the specific activity of \atpdx~ for converting I320 to PLP in the presence of G3P. Figures \ref{fig:pdx1_WT_ts_320_fit} and \ref{fig:pdx1_WT_ts_PLP_fit} demonstrate that formation of the I320 intermediate and PLP occurs linearly in the time periods chosen, the R-square values for all wild type assays were greater than 0.95. The extinction coefficient of PLP at 415 nm is 5,380 \cite{Raschle2007}. The specific activity of \atpdx ~for conversion of I320 to PLP is 0.47 \act ~with a standard deviation of 0.09 \act. %\textit{B. subtilis} Pdx1 also showed a faster rate of catalysis for formation of I320 than for conversion of I320 to PLP \cite{Raschle2007}.    

   



\clearpage
\section{Crystal structures of wild type \atpdx}

The structure of the wild type \atpdx~ protein was solved in several different intermediate states to improve our understanding of the catalytic mechanism of Pdx1. \atpdx ~was crystallised and soaked with substrates as described in Section \ref{sec:Crystallisation_Methods}. While crystal structures of Pdx1 in the R5P and PLP bound states have been published \cite{Guedez2012,Smith2015}, the mechanism by which the biosynthetic reaction is transferred from the P1 site to the P2 site was unknown. The structures presented in this section allow us to characterise the intermediates between the R5P and PLP bounds states, and elucidate the mechanism of intermediate transfer between the P1 and P2 sites.

\subsection{\atpdx -R5P}
The \atpdx -R5P structure was solved by Dr Ivo Tews using data collected from protein crystals produced by Stefan Weber (unpublished). The crystal structure shows that C1 of R5P binds covalently to the $\varepsilon$-nitrogen of the lysine 98 (Figure \ref{fig:Pdx1_R5P}). The phosphate group of R5P is bound in the P1 site, as is observed in the published structures of \textit{Pb}Pdx1 and \textit{Gs}Pdx1 in complex with R5P \cite{Guedez2012,Smith2015}. The phosphate group is bound in the P1 site by hydrogen bonds to N-H groups on the polypeptide backbone from residues Glycine 170, Glycine 231, Glycine 252 and Serine 253 (Figure \ref{fig:Pdx1_R5P_P1}). 

The R5P in the soaking solutions is in the furanose form with the five carbon atoms forming a closed ring (Figure \ref{fig:Pdx1_R5P}, \textcolor{red}{1}); Figure \ref{fig:Pdx1_R5P} shows the open form of R5P bound to Pdx1. The opening of the furanose ring precedes the formation of the covalent complex observed in the Pdx1-R5P structure (Figure \ref{fig:Pdx1_R5P}), less than 1\% of R5P in solution is present in the open form, and it is unknown whether Pdx1 selectively binds the open form or catalyses ring opening \cite{Pierce1984}. A possible mechanism for binding of R5P to Pdx1 is shown in Figure \ref{fig:R5P_isomers} and is based on the reaction mechanism proposed by Hanes \textit{et al} \cite{Hanes2008b}. The isomer of R5P modelled as being bound to Pdx1 in the crystal structure is compound \textcolor{red}{4} in Figure \ref{fig:R5P_isomers}.     


% Opening of the furanose ring precedes formation of the covalent bond between C1 and Lys98, as less than 1\% of R5P in solution is present in the open form and Pdx1 is capable of using R5P or ribulose 5-phosphate (Ru5P)\glsadd{Ru5P} as the pentose substrate it has been suggested that Pdx1 binds R5P in the ring form and catalyses the isomerisation of R5P to Ru5P (Figure \ref{fig:R5P_isomers})\cite{Pierce1984,Burns2005,Raschle2007}. 

The observation that the oxygen atoms on carbons 2, 3, and 4 remain bound to R5P in the Pdx1-R5P complex is consistent with previous NMR experiments on the complex formed between Pdx1 and $^{13}$C labelled R5P \cite{Hanes2008b}. The oxygen atom bound to carbon 1 is eliminated during formation of C1-R5P bond. Residues Aspartate 41, Aspartate 119 and Serine 121 form hydrogen bonds with the covalently bound intermediate and may participate in catalysing the formation of the Lys 98-R5P complex (Figure \ref{fig:Pdx1_R5P}). The Pdx1-R5P complex was modelled as \textcolor{red}{4} rather than \textcolor{red}{3}, as the published NMR experiments on the Pdx1-R5P complex identified that the C2 oxygen is bound as a ketone rather than as a secondary alcohol \cite{Hanes2008b}. 
 
\begin{minipage}{\linewidth}
	\makebox[\linewidth]{
	\includegraphics[width=10cm, keepaspectratio]{/Users/matt/Dropbox/ThesisPrep/Thesis/fig/sw5-4/sw5_4hbond2.png}}	
	\captionof{figure}[Crystal structure of the wild type \atpdx~ active site in the R5P bound state]{Crystal structure of wild type \atpdx~ with R5P bound, at a resolution of 1.91 \si{\angstrom}. The sidechains of Aspartate 41, Lysine 98, Aspartate 119 and Serine 121 are shown (carbon atoms forest, nitrogen atoms cyan, oxygen atoms red). R5P binds to the protein and forms a covalent bond between C1 and Lys 98 (intermediate carbon atoms orange, oxygen atoms red, phosphorous atoms purple). 2Fo-Fc electron density map is contoured around the lysine residue 98 and R5P intermediate at 1 $\sigma$.\label{fig:Pdx1_R5P}} 		
\end{minipage}   

\begin{minipage}{\linewidth}
	\makebox[\linewidth]{
	\includegraphics[width=10cm, keepaspectratio]{/Users/matt/Dropbox/ThesisPrep/Thesis/fig/sw5-4/P1/P1_hydrogenbonds_sw5-4_label}}	
	\captionof{figure}[Co-ordinaton of the R5P phosphate by the P1 binding site]{The hydrogen bonds (red) between the P1 phosphate binding site and the phosphate group of R5P. The nitrogen atoms of the peptide backbone, for residues contributing to phosphate binding, are shown as blue spheres. \label{fig:Pdx1_R5P_P1}} 		
\end{minipage}       

\begin{minipage}{\linewidth}
	\makebox[\linewidth]{
	\includegraphics[width=16cm, keepaspectratio]{/Users/matt/Dropbox/ThesisPrep/Thesis/fig/reactionsceme/sw5_4chemdraw4.pdf}}	
	\captionof{figure}[Reaction steps leading to formation of Pdx1-R5P]{Ribose 5-phosphate is typically observed in the closed ring form in solution (\textcolor{red}{1}), before the formation of the Pdx1-R5P complex R5P isomerises to the open chain form (\textcolor{red}{2}). The Lys 98 $\varepsilon$-NH$_2$ - C1 imine bond is formed in a dehydration reaction (\textcolor{red}{3}). R5P then tautomerises from the C2 hydroxyl form to the ketone form while the $\varepsilon$-NH$_2$ becomes protonated (\textcolor{red}{4}). This reaction mechanism is based on the mechanism proposed by Hanes \textit{et al} \cite{Hanes2008b}. \label{fig:R5P_isomers}} 		
\end{minipage}  
\clearpage
\subsection{\atpdx -I320}
The crystal structure of \atpdx -I320 was first solved by Dr Volker Windeisen in the Tews group (unpublished). Initial attempts to reconstitute I320 formation \textit{in crystallo} yielded crystals that were partially occupied in the I320 state and partially in the R5P bound state; this conformational heterogeneity made interpretation of the electron density in the active site difficult. The structure presented here was a repeat of the initial experiment, with the crystallisation and soaking protocols optimised to obtain a dataset at a higher resolution and with reduced conformational heterogeneity.\par

UV-Vis spectra were collected from the crystals before data collection, to ensure that the protein was in the I320 state (Figure \ref{fig:I320_xtal_spec}). The spectra resembled those of \atpdx~ in solution after an I320 assay (Figure \ref{fig:I320_xtal_spec}), suggesting that most of the protein was in the I320 state.

\begin{minipage}{\linewidth}
	\makebox[\linewidth]{
	\includegraphics[width=6cm, height=6cm, keepaspectratio]{/Users/matt/Dropbox/ThesisPrep/Thesis/fig/spectra/crystal/I320.pdf}}	
	\captionof{figure}[UV-Vis spectrum of an \atpdx -I320 crystal]{UV-Vis spectrum of an \atpdx -I320 crystal, the spectrum was smoothed using a Savitsky-Golay filter in MATLAB (Appendix \ref{App:MATLABSPEC_SGOLAY}). The spectrum was collected using the online microspectrophotometer at ESRF beamline ID14-1.\label{fig:I320_xtal_spec}} 		
\end{minipage}

The crystal structure shows that the intermediate remains covalently bound to Lys 98 via C1 from R5P (Figure \ref{fig:Pdx1_I320}). As oxygen and nitrogen have similar atomic numbers, their atomic scattering factors are also similar; this makes it difficult to discern whether C2 of the intermediate is bound to an oxygen atom or a nitrogen atom, purely based on the X-ray diffraction data. In cases where the crystal diffracts to better than 1\si{\angstrom} resolution, it is possible to distinguish between carbon, nitrogen and oxygen by inspection of the electron density alone, but it is not possible with the 1.7 \si{\angstrom} \atpdx -I320 dataset \cite{Betzel2001}. However, previous NMR experiments using $^{13}$C labelled R5P and $^{15}$N labelled ammonia to reconstitute the Pdx1-I320 complex, have shown that nitrogen incorporation at the C2 position precedes the formation of the I320 intermediate \cite{Hanes2008b}. As the spectra of the crystals indicated that I320 formation had occurred (Figure \ref{fig:I320_xtal_spec}), the atom bound to C2 has been modelled as nitrogen.  

The published NMR experiment with $^{13}$C labelled I320 identified that the C3 bound oxygen is the only oxygen atom to remain from R5P \cite{Hanes2008b}. Consistent with published mass spectrometry experiments on the I320 intermediate, the phosphate group originating from R5P is eliminated before I320 formation and is observed to remain bound in the P1 site (Figure \ref{fig:Pdx1_I320}) \cite{Hanes2008a}. Mass spectrometry experiments were used to determine that in the transition from Pdx1-R5P to Pdx1-I320, the phosphate group is eliminated, nitrogen incorporation occurs and two hydroxyl groups, observed here to be the C2 and C4 hydroxyl groups, are removed in dehydration reactions \cite{Hanes2008b,Hanes2008a}.   
  
\begin{minipage}{\linewidth}
	\makebox[\linewidth]{
	\includegraphics[width=10cm, height=10cm, keepaspectratio]{/Users/matt/Dropbox/ThesisPrep/Thesis/fig/ma4-1/ma4-1hbond160512label.png}}	
	\captionof{figure}[Crystal structure of the wild type \atpdx~ active site in the I320 state]{Crystal Structure of wild type \atpdx ~solved at 1.7 \si{\angstrom}. The sidechains of Aspartate 41, Lysine 98, Serine 121 and Lysine 166 are shown in stick format (carbon atoms forest, nitrogen atoms cyan, oxygen atoms red). The covalent I320 intermediate links the two lysine side chains (carbon atoms orange, nitrogen atoms blue, oxygen atoms red). 2Fo-Fc electron density map is contoured around the lysine residues and I320 intermediate at 1 $\sigma$.\label{fig:Pdx1_I320}}
\end{minipage}

In contrast to previously proposed structures for the I320 intermediate, we observe that Lysine 166 orients towards the intermediate and a covalent bond is formed between C5 of the intermediate and the Lysine 166 $\varepsilon$-NH$_2$ (Figure \ref{fig:Pdx1_I320}). In the R5P bound state Lys 166 is oriented towards the P2 site rather than the P1 site, this conformational change occurs as a result of a peptide flip between Lysine 166 and Glycine 167 on the $\beta$ strand 6 (Figure \ref{fig:I320_Beta6}). The ability of Lys 166 to re-orient and participate in catalysis in the P1 site was first proposed by Zhu \textit{et al} on the basis that the residues either side of Lys 166 on $\beta$6 lack hydrogen bonds to adjacent strands \cite{Zhu2005}. 

The peptide flip appears to be coupled to the addition of ammonia to \atpdx -R5P; Met 162 is one of the conserved methionine residues that line the transient tunnel that ammonia is channelled through, and is located on strand $\beta$6 (Figure \ref{fig:I320_Met}). It is possible that the passage of ammonia through the tunnel disrupts the conformation of the $\beta$6 strand via Met 162 and through this mechanism couples glutamine hydrolysis to the conformational changes in Pdx1 required for I320 formation. A mutagenesis study by Guedez \textit{et al} identified the equivalent residue to Methionine 162 in \textit{Plasmodium} Pdx1 as being important for coupling of the Pdx2 glutaminase activity to the biosynthesis of PLP \cite{Guedez2012}. 

\begin{figure}[!htbp]
\centering
\begin{subfigure}{.49\textwidth}
  \centering
  \includegraphics[width=8cm, keepaspectratio]{/Users/matt/Dropbox/ThesisPrep/Thesis/fig/ma4-1/betaoverlay/I320_R5Poverlay_arrow}
  \caption{\label{fig:I320_Beta6}}
\end{subfigure}
\begin{subfigure}{.49\textwidth}
  \centering
  \includegraphics[width=8cm, keepaspectratio]{/Users/matt/Dropbox/ThesisPrep/Thesis/fig/ma4-1/ma4-1_Methionines}
  \caption{\label{fig:I320_Met}}
\end{subfigure}
\caption[Conformational changes facilitating the R5P - I320 Transition]{(a) The conformational changes that occur on strand $\beta$6 in the R5P (red outline) - I320 (purple outline) transition. (b) Ammonia must pass from the Pdx1 - Pdx2 interface site at the bottom of the \TIM ~barrel, through the hydrophobic tunnel gated by Methionine 92 and Methionine 162, to reach the site of incorporation in I320.}
\end{figure}  

The published NMR experiments showed that both C1 and C5 of I320 were bound to nitrogen atoms; however, this was interpreted as being caused by the formation of an off-pathway compound as a result of I320 cyclizing under the denaturing conditions used for the experiment \cite{Hanes2008b}. This crystal structure shows that the NMR data was collected from an intact I320 complex and that the nitrogen atoms were those of the two lysine side chains.      

A summary of the reaction that occurs in the R5P - I320 transition is presented in Figure \ref{fig:R5PtoI320}. It is known that addition of ammonia to Pdx1-R5P triggers phosphate elimination, dehydration at the C4 position and substitution of the C2 hydroxyl for an amino group \cite{Hanes2008a,Hanes2008b}. The intermediate states that the reaction passes through between the R5P and I320 states have not been structurally characterised; the use of site-directed mutagenesis may allow the trapping of the enzyme in these intermediate states.       

%By combining the information provided by this crystals structure with published data on the transition from the R5P bound state to I320 it is possible to postulate an updated mechanism for formation of the I320 intermediate (Figure \ref{fig:R5PtoI320}). The intermediate states in Figure \ref{fig:R5PtoI320} for which there is experimental evidence are the Pdx1-R5P complex (\textcolor{red}{4}) and the Pdx1-I320 complex (\textcolor{red}{9}). While it is known that phosphate elimination and C4 hydroxyl group occur in the transition from \textcolor{red}{4} to \textcolor{red}{9}, the mechanism by which this occurs is hypothetical.  

\begin{minipage}[!htbp]{\linewidth}
	\makebox[\linewidth]{
	\includegraphics[width=16cm, keepaspectratio]{/Users/matt/Dropbox/ThesisPrep/Thesis/fig/reactionsceme/R5PtoI320_Simple2.pdf}}	
	\captionof{figure}[Reaction scheme for the R5P - I320 transition]{The reaction scheme for the transition from the Pdx1-R5P state to the Pdx1-I320 state.\label{fig:R5PtoI320}} 		
\end{minipage}  
\clearpage
%
%\begin{minipage}[!htbp]{\linewidth}
%	\makebox[\linewidth]{
%	\includegraphics[width=16cm, keepaspectratio]{/Users/matt/Dropbox/ThesisPrep/Thesis/fig/reactionsceme/R5PtoI320_BA_2gimp.pdf}}	
%	\captionof{figure}[Reaction scheme for Pdx1-R5P to Pdx1-I320 Transition]{The Pdx1-R5P complex (\textcolor{red}{4}) is stable until the addition of ammonia. The nitrogen of ammonia is incorporated at the C2 position in a substitution reaction (\textcolor{red}{5}). A proton is donated by an unspecified acid (A-H) to the C4 hydroxyl group, which is eliminated in a dehydration reaction to form \textcolor{red}{6}. The C5 atom of \textcolor{red}{6} is deprotonated, allowing the covalent intermediate to tautomerise to form \textcolor{red}{7}. The \nh ~of Lys 166 reacts with C5 of \textcolor{red}{7} and donates a proton to the C3 oxygen to form \textcolor{red}{8}. The C3 oxygen of \textcolor{red}{8} is deprotonated by an unspecified base in the active site (B), resulting in elimination of the phosphate group and formation of the I320 intermediate, \textcolor{red}{9}.    \label{fig:R5PtoI320}} 		
%\end{minipage}  
%\clearpage
%

\subsection{\atpdx -I320/G3P}
The structure of the \atpdx -I320/G3P complex presented here was solved using data collected by Dr Volker Windeisen and refined by Dr Yang Zhang (University of Cornell) \cite{Windeisen2013}. We observe that G3P binds in the P1\glsadd{P1}\glsadd{P2} site and that the phosphate group of G3P can displace the phosphate group eliminated from R5P during formation of the I320 intermediate. G3P binds covalently to the nitrogen atom of I320 that was incorporated from ammonia during I320 formation (Figure \ref{fig:Pdx1_G3P}). %The condensation reaction between carbon 1 of G3P and the C2 nitrogen of I320 results in the loss of one molecule of water. %The covalent attachment of G3P to I320 leads to the release of the intermediate from the \nh ~of Lys 98, allowing Lys 166 to re-orient towards the P2 site where PLP biosynthesis is completed.

\begin{minipage}{\linewidth}
	\makebox[\linewidth]{
	\includegraphics[width=10cm, keepaspectratio]{/Users/matt/Dropbox/ThesisPrep/Thesis/fig/vw47-2/Pdx1_G3Plabel.png}}	
	\captionof{figure}[Crystal structure of wild type \atpdx~ in the I320/G3P intermediate state]{Crystal structure of wild type \atpdx~ in the I320 state with G3P bound at 1.90 \si{\angstrom} resolution. The sidechains of Aspartate 41, Lysine 98 and Serine 121 are shown in stick format (carbon atoms forest, nitrogen atoms cyan, oxygen atoms red). The I320 intermediate forms a covalent link between Lys 98 and Lys166 (I320 carbon atoms orange, oxygen atoms red, phosphorous atoms purple, carbon atoms originating from G3P shown in yellow). 2Fo-Fc electron density map is contoured around the covalent intermediate and lysine residues 98 and 166 at 1 $\sigma$.\label{fig:Pdx1_G3P}} 		
\end{minipage}     

Until this structure was solved, it was unknown whether G3P bound to Pdx1 in the P1 or P2 site. The P1 site of Pdx1 is located in a similar position on the ($\beta$/$\alpha$)$_{8}$ barrel as the active site for the archetypal ($\beta$/$\alpha$)$_{8}$ protein, Triose Phosphate Isomerase, which binds G3P and catalyses its isomerisation to dihydroxyacetone phosphate \cite{Noble1991}. 

There is also electron density in this structure for a ligand bound in the P2 site (Figure \ref{fig:P2_R5P} \& \ref{fig:P2_G3P}). The electron density suggests that the molecule is phosphorylated; the two compounds most likely to be binding in the P2 site are, therefore, R5P and G3P, as they are both present in the soaking solutions. Refinement with G3P or R5P fitted into the density individually produced difference density peaks around the compounds, suggesting that the models were not correct. Refinement with both R5P and G3P fitted into the density, with occupancies of 0.66 and 0.34 respectively, produced no significant difference density peaks. This suggests that the binding of these compounds in the P2 site may be non-specific, and is primarily due to the affinity of the P2 site for the phosphate groups. The observation that the furanose ring of R5P only hydrogen bonds to water molecules, while the phosphate group forms several hydrogen bonds to the sidechains of Arginine 154 and Arginine 155 on helix $\alpha$6, and Lysine 204 from a neighbouring subunit, supports this hypothesis (Figure \ref{fig:P2_R5P_bonds}).         

\begin{figure}[!htbp]
\centering
\begin{subfigure}{.3\textwidth}
  \centering
  \includegraphics[width=5.5cm, height=5.5cm, keepaspectratio]{/Users/matt/Dropbox/ThesisPrep/Thesis/fig/vw47-2/P2/R5P_chain_P2.png}
  \caption{}
  \label{fig:P2_R5P}
\end{subfigure}%
\begin{subfigure}{.3\textwidth}
  \centering
  \includegraphics[width=5.5cm, height=5.5cm, keepaspectratio]{/Users/matt/Dropbox/ThesisPrep/Thesis/fig/vw47-2/P2/G3P_chainB_P2.png}
  \caption{}
  \label{fig:P2_G3P}
\end{subfigure}
\begin{subfigure}{.3\textwidth}
  \centering
  \includegraphics[width=5.5cm, height=5.5cm, keepaspectratio]{/Users/matt/Dropbox/ThesisPrep/Thesis/fig/vw47-2/P2/R5P_chain_P2_spheres_gimp.png}
  \caption{}
  \label{fig:P2_R5P_bonds}
\end{subfigure}%
\caption[P2 site ligands in the Pdx1 I320/G3P Structure]{The P2 site in the Pdx1-I320/G3P structure shows electron density suggesting ligand binding. (a) R5P and (b) G3P were fitted into the density with occupancies of 0.66 and 0.34 respectively. (c) The phosphate group of the P2-bound R5P forms hydrogen bonds with the sidechains of Arginine 154, Arginine 155 and Lysine 204 of a neighbouring subunit. Protein carbon atoms forest, protein nitrogen atoms cyan, ligand carbon atoms orange, ligand oxygen atoms red, ligand phosphorous atoms purple. Electron density displayed as a 2Fo-Fc map contoured at 1$\sigma$.}
\end{figure}

The observation that G3P binds in the P1 site and reacts covalently with I320 before the Lysine 98 releases the intermediate is not consistent with the order of events presented in the mechanism of PLP biosynthesis published by Hanes \textit{et al} \cite{Hanes2008b}. The electron density for the covalent intermediate shows that the C1 bound oxygen atom of G3P is retained at this stage in catalysis (Figure \ref{fig:Pdx1_G3P}). This suggests that the formation of the covalent bond between the C2 nitrogen of I320 and C1 of G3P occurs in a nucleophilic addition reaction. A possible mechanism by which this may occur is presented in Figure \ref{fig:I320toG3P}. The aldehyde group of G3P (\textcolor{orange}{1}) is protonated, either by the solvent or a catalytic residue. After the protonated G3P binds in the P1 site, there is a nucleophilic attack from the lone pair of the I320 C2 nitrogen to the C1 of G3P, resulting in the addition of the two compounds and formation of \textcolor{orange}{3}. The charged nitrogen originating from I320 is then deprotonated, either by the solvent or a catalytic residue to form \textcolor{orange}{4}, which is the compound that has been modelled as the I320/G3P intermediate in Figure \ref{fig:Pdx1_G3P}.  

As C1 of G3P becomes C6 of PLP, which is not bound to a hydroxyl group, this group must by eliminated before product formation. C2 of G3P becomes C5 of PLP, which also lacks a hydroxyl group, this group must also be eliminated in the final steps of PLP biosynthesis. Once the I320/G3P complex is formed, all of the atoms required to form PLP are present; and only release of the intermediate from Lysine 98, closure of the pyridine ring and aromatisation have to take place before the product is formed.    
      

 %Figure \ref{fig:I320toG3P} depicts a possible mechanism for the covalent reaction of G3P with the I320 intermediate. The mechanism is based on the reaction proceeding as a nucleophilic addition of the I320 C2 amine group to the C1 aldehyde of G3P followed by elimination of a molecule of water. G3P exists in an equilibrium between having an uncharged and positively charged C1 carbonyl group, in the protonated form C1 becomes susceptible to nucleophilic attack by the C2 amine group of I320 (\textcolor{red}{9}). The nucleophilic attack causes one of the electrons from the carbonyl pi bond to shift to the oxygen atom, neutralising the positive charge, and the other is required for formation of the nitrogen - carbon sigma bond in intermediate \textcolor{red}{10} (Figure \ref{fig:I320toG3P}). The C2 nitrogen atom from I320 is then deprotonated, either by a water molecule or a catalytic residue to form \textcolor{red}{11}. Protonation of the C1 hydroxyl group of \textcolor{red}{11} allows the hydroxyl group to be eliminated as a water molecule and formation of an imine between the C2 nitrogen and C1 of G3P (\textcolor{red}{12}). The charged nitrogen atom may be deprotonated by a catalytic residue or a water molecule from the solvent, to form \textcolor{red}{13}, which is modelled in Figure \ref{fig:Pdx1_G3P}.                 

\begin{minipage}{\linewidth}
	\makebox[\linewidth]{
	\includegraphics[width=16cm, keepaspectratio]{/Users/matt/Dropbox/ThesisPrep/Thesis/fig/reactionsceme/I320toG3P_2gimp.pdf}}	
	\captionof{figure}[Proposed mechanism for addition of G3P to I320]{A possible mechanism for the addition of G3P to the I320 intermediate to form the Pdx1-I320/G3P intermediate (\textcolor{orange}{4}).\label{fig:I320toG3P}} 		
\end{minipage}      

Enzyme assays show that in solution, the addition of G3P to \atpdx -I320 triggers a reduction in the magnitude of the I320 peak and an increase in absorbance at $\sim$415 nm for PLP. The reason that it was possible to trap the enzyme in this state is unclear; however, it is possible that the packing of the protein in the crystal lattice prevents the conformational changes required  for the transition from the I320/G3P bound to the product state from taking place. It is known that the C-terminus of Pdx1, which is disordered in crystal structures, interacts with the active site of the neighbouring subunit and is required for conversion of I320 to PLP \cite{Raschle2009,Moccand2011}. The C-terminus may be restricted from adopting the conformation necessary for catalysis, by the crystal lattice, resulting in the trapping of the I320/G3P intermediate.    

\clearpage      
\subsection{\atpdx -PLP}
The Pdx1-PLP complex has been studied previously with X-ray crystallography, NMR and mass spectrometry. Mass spectrometry was used to identify Lysine 166 (Lysine 149 in \textit{B. subtilis}) as the site of covalent binding of PLP by Pdx1 \cite{Moccand2011}. NMR experiments on Pdx1-PLP using $^{13}$C labelled R5P to reconstitute the complex identified that the product remains bound to the $\varepsilon$-NH$_2$ of the lysine residue by an imine bond \cite{Hanes2008b}. However, the only published crystal structure of Pdx1 in complex with PLP showed PLP bound in the P2 site without a covalent link tot he protein \cite{Zhang2010}. The aim of this experiment was to determine how PLP binds to \atpdx .


Attempts to reconstitute the formation of PLP \textit{in crystallo}, using R5P, ammonia and G3P were unsuccessful; most likely because of the constraints imposed on the mobility of the Pdx1 C-terminus by the crystal lattice. The structure of \atpdx~ in complex with PLP was determined by performing crystallography experiments with \atpdx~ crystals soaked with PLP, as described in Section \ref{sec:PLP_soak}. The experiment described here used a protocol originally developed by Marco Strohmeier in the Tews group when reconstituting the \textit{Bacillus subtilis} Pdx1-PLP complex (unpublished). 

In addition to X-ray diffraction data, UV-Vis spectra were collected from \atpdx ~crystals after soaking with PLP. The UV-Vis spectrum shows an absorbance peak with a \lwl = 415 nm; PLP in the free form has an absorbance maximum at 388.5 nm, while PLP covalently bound to protein has an absorption maximum at 415 nm (Figure \ref{fig:Crystal_PLP_Spec}) \cite{Hanes2008b}. The spectrum, therefore, suggests that PLP is covalently bound to Pdx1 in the crystal.   

\begin{minipage}{\linewidth}
	\makebox[\linewidth]{
	\includegraphics[width=6cm, keepaspectratio]{/Users/matt/Dropbox/ThesisPrep/Thesis/fig/ma17-3/Pdx1_PLPxtalspec.pdf}}	
	\captionof{figure}[UV-Vis spectrum of an \atpdx -PLP crystal]{UV-Vis spectrum of an \atpdx ~crystal after soaking with PLP and cryocooling to 100 K. The spectrum was smoothed using a Savitzky-Golay filter (Section \ref{sec:SPEC_PROC}).\label{fig:Crystal_PLP_Spec}}		
%%% Replace figure to show origin of cabon atoms and final numbering
\end{minipage}

The structure shows that PLP binds covalently to Lysine 166, which has re-oriented towards the P2 site, published mass spectrometry experiments had previously identified Lys 166 and the P2 site as the probable PLP binding site (Figure \ref{fig:Pdx1_PLP}) \cite{Moccand2011}. The phosphate group of PLP is anchored in the P2 site by hydrogen bonds to the sidechains of residues Histidine 132 on helix $\alpha$4, Arginine 154 and Arginine 155 on helix $\alpha$6, and the $\varepsilon$-NH$_2$ of Lysine 204 from the neighbouring subunit on the opposite hexamer (Figure \ref{fig:PLP_P2}). 

Carbons 2, 2', 3, 4 and 4' of PLP originate from what were carbon atoms 1, 2, 3, 4, and 5 of R5P respectively (Figure \ref{fig:G3PtoPLP}). Carbons 5, 5', 6 and the phosphate group of PLP originate from carbon atoms 2, 3 and 1 of G3P respectively (Figure \ref{fig:G3PtoPLP}). The nitrogen atom is derived from ammonia which is produced by Pdx2-catalysed glutamine hydrolysis. The bond between PLP and Lysine 166 is formed between the C4’ atom of PLP, which was originally C5 of R5P, and the $\varepsilon$-NH$_2$ of Lysine 166. As mentioned previously, the bond between Pdx1 and PLP has been characterised as an imine \cite{Hanes2008b}. Pdx1, therefore, forms an internal aldimine with PLP (Figure \ref{fig:transimination}), with the same structure as would be observed in a PLP-dependent enzyme.


The elimination of two molecules of water and the closure of the pyridine ring occur in the I320/G3P - PLP transition. Given that several reactions must occur in this transition, and that none of the intermediates between the two states have been characterised, it is not possible to produce a detailed chemical mechanism for the transition that is based on experimental evidence. Instead, a summary of the reaction is provided in Figure \ref{fig:G3PtoPLP}. It is unknown whether ring closure and aromatisation occur in the P1 site or the P2 site, but Figure \ref{fig:Pdx1_PLP} does show that the product remains covalently bound to the enzyme in the P2 site.    


%The closure of the pyridine ring is likely to be dependent on the C2 atom of G3P and C4 of R5P being brought in to close enough proximity for a condensation reaction to occur.  %The hydroxyl group originally bound to C2 of G3P is likely to be     

%In order to transition from the P1 to the P2 site the bond between Lys98 and the carbon that was initially C1 of R5P observed in the \atpdx -I320/G3P structure must be cleaved while the Lys 166 bond to the R5P carbon 5 remains intact. Closure of the pyridine ring must take place for synthesis of PLP to be completed, the hydroxyl group originating from C2 of G3P is eliminated in a dehydration reaction during ring closure. On the basis of the structures presented here it is unclear whether closure of the pyridine ring occurs in the P1 site before Lys 166 re-orients towards P2 or if it takes place in the P2 site.

\begin{minipage}{\linewidth}
	\makebox[\linewidth]{
	\includegraphics[width=10cm, keepaspectratio]{/Users/matt/Dropbox/ThesisPrep/Thesis/fig/ma17-3/ma17-3cartoonhbond160512label.png}}	
	\captionof{figure}[Crystal structure of the wild type \atpdx~ active site with PLP bound]{Crystal structure of wild type \atpdx~ solved at 1.61 \si{\angstrom}. The sidechains of Lysine 98 and Lysine 166 are shown in stick format (carbon atoms forest, nitrogen atoms cyan). PLP is covalently bound to Lysine 166 (carbon atoms originating from R5P orange, carbon atoms originating from G3P yellow, nitrogen atoms blue, oxygen atoms red, phosphorous purple). 2Fo-Fc electron density map is contoured around Lysine 166 and PLP at 1 $\sigma$.\label{fig:Pdx1_PLP}} 		
%%% Replace figure to show origin of carbon atoms and final numbering
\end{minipage}
\\
\begin{minipage}{\linewidth}
	\makebox[\linewidth]{
	\includegraphics[width=6cm, keepaspectratio]{/Users/matt/Dropbox/ThesisPrep/Thesis/fig/ma17-3/P2_PLP/P2_PLPgimp.png}}	
	\captionof{figure}[Coordination of PLP phosphate in the Pdx1 P2 site]{The phosphate group of PLP is coordinated by the sidechains of Histidine 132, Arginine 154, Arginine 155 in the P2 site, in addition to the sidechain of Lysine 204 from a neighbouring subunit. Atoms coloured as described in Figure \ref{fig:Pdx1_PLP}.\label{fig:PLP_P2}}		
\end{minipage}

\begin{minipage}{\linewidth}
	\makebox[\linewidth]{
	\includegraphics[width=15cm, keepaspectratio]{/Users/matt/Dropbox/ThesisPrep/Thesis/fig/reactionsceme/G3PtoPLPsimple.pdf}}	
	\captionof{figure}[Summary reaction scheme for transition from I320/G3P to PLP]{A summary of the reaction that occurs during the conversion of the I320/G3P intermediate (\textcolor{orange}{4}) to the product, PLP (\textcolor{orange}{6}). The intermediate is released from Lysine 98 and two water molecules are released during closure of the pyridine ring and aromatisation to convert \textcolor{orange}{4} to the Pdx1-PLP complex observed in Figure \ref{fig:Pdx1_PLP} (\textcolor{orange}{5}). The imine between Lys 166 and PLP must be hydrolysed before the product can be released. The labelling of the atoms in \textcolor{orange}{4} and \textcolor{orange}{5} relates to the original numbering in the substrates, the labelling of the atoms in \textcolor{orange}{6} relates to the numbering of the atoms in PLP.\label{fig:G3PtoPLP}}		
\end{minipage}


 
%Figure \ref{fig:G3PtoPLP} shows a possible mechanism for the transition from the I320/G3P state (\textcolor{red}{13}) observed in Figure \ref{fig:Pdx1_G3P} to the product state (\textcolor{red}{19}). For the reaction to transition to the P2 site the bond between Lys 98 and the atom that was orginally C1 of R5P must be cleaved. This may be achieved by protonation of the Lys 98 \nh~ and deprotonation at the C2 position of I320 to form \textcolor{red}{14}, resonance of the double bonds in \textcolor{red}{14} results in cleavage of the bond to Lys 98 and formation of \textcolor{red}{15}. Tautomerisation of \textcolor{red}{15} to \textcolor{red}{16} and \textcolor{red}{17} may be catalysed by active site residues or enabled by interaction with the solvent, resulting in conversion of the amide bond between \textcolor{red}{17} and Lys 166 to an imine. Hanes \textit{et al} determined that the product is bound to the lysine residue via an imine using NMR \cite{Hanes2008b}. Ring closure of \textcolor{red}{17} to form \textcolor{red}{18} is followed by elimination of a molecule of water to form the product, PLP (\textcolor{red}{19}), which is bound to Lys 166 by an imine bond.  
%\begin{minipage}{\linewidth}
%	\makebox[\linewidth]{
%	\includegraphics[width=16cm, keepaspectratio]{/Users/matt/Dropbox/ThesisPrep/Thesis/fig/reactionsceme/G3PtoPLPgimp.png}}	
%	\captionof{figure}[Mechanism for the conversion of I320/G3P to PLP]{A possible mechanism for the formation of PLP following the reaction of G3P with I320 to form the Pdx1-PLP complex (\textcolor{red}{19}) observed in Figure \ref{fig:Pdx1_PLP}.\label{fig:G3PtoPLP}} 		
%\end{minipage}  
\clearpage
\section{Discussion of wild type \atpdx~ results}

The results presented in the chapter demonstrate that it is possible to reconstitute the activity of \atpdx~ in solution, and to an extent, \textit{in crystallo}. By integrating the biochemical and structural data presented in this chapter, it has been possible to characterise some of the intermediates of Pdx1-catalysed PLP biosynthesis. The four structures presented in this chapter also clarify the mechanism by which Pdx1 transfers reaction intermediates from the P1 site to the P2 site (Figure \ref{fig:cartoon_panel}). 

The key finding presented in the chapter is that the I320 intermediate is simultaneously bound to  bound to both Lysine 98 and Lysine 166 (Figure \ref{fig:I320_panel}). The addition of ammonia to \atpdx -R5P appears to trigger the conformational change in strand $\beta$6 that results in the reaction of Lys 166 with C5 of R5P and the elimination of the R5P phosphate group. The dual linkage of the enzyme to the intermediate allows Pdx1 to maintain control of the position of the atoms of the I320 intermediate once the phosphate of R5P has been eliminated. The elimination of the R5P phosphate group must precede binding of G3P, as the phosphate of G3P binds in the P1 site of Pdx1 and displaces the R5P phosphate (Figure \ref{fig:cartoon_panel}). Maintaining control of the position of the I320 intermediate ensures that when G3P binds, the C2 nitrogen of I320 is in the correct position to react with the aldehyde group of G3P.      

The order of events that occur between the addition of G3P to I320 and the formation of the product remain unknown. We do now know that release of the intermediate from Lysine 98 occurs after the reaction of G3P with I320, but it is unknown whether release from Lysine 98 precedes or follows ring closure and aromatisation. What is clear, is that in the transition from the I320/G3P state to the product state, the phosphate group from G3P must be released from the P1 site and the covalent bond to Lysine 98 must be cleaved. The requirement to disrupt these interactions places an activation energy barrier between the states and may be rate limiting. 

We observe that PLP remains covalently bound to Pdx1 via an imine bond between C4' of PLP and the $\varepsilon$-NH$_2$ of Lysine 166 (Figure \ref{fig:PLP_panel}). The millimolar concentrations of PLP present in the crystal soaking solutions ensured that the protein was fully occupied in the PLP bound state. A significant fraction of the recombinantly expressed \atpdx ~also purified with PLP bound to the protein (Figure \ref{fig:Pdx1WT_RayleighSub}), suggesting that there may be some product inhibition of the enzyme. 

Exchange of catalytic residues involved in PLP biosynthesis may enable trapping of further intermediates not accessible with the wild type protein. The experiments performed on variants of \atpdx ~with point mutations made to residues expected to contribute to catalysis have been described in chapter \ref{ch:Mutant_Results}.   

%The crystal structure of Pdx1-R5P clearly shows that R5P binds covalently to Lys 98 in the P1 site of Pdx1 which is consistent with previous crystal structures of the Pdx1-R5P complex and with mass spectrometry data (Figure \ref{fig:Pdx1_R5P}) \cite{Guedez2012,Raschle2007,Zein2006}. The C1 atom of R5P is observed to form the linkage with Lys 98, this is consistent with the NMR data for the denatured Pdx1-R5P complex and the \textit{Plasmodium falciparum} Pdx1 crystal structure \cite{Hanes2008b,Guedez2012}. Prior to formation of a covalent bond to the protein R5P must isomerise from the closed ring form to the open chain and formation of the imine bond between R5P and Lys 98 proceeds via a condensation reaction (Figure \ref{fig:R5P_isomers}).

%The observation that the I320 intermediate forms a covalent bridge between the two active site lysine residues was surprising considering the absence of similar structures in the literature. However, the model proposed for the intermediate structure is consistent with the previously published NMR data collected from the denatured \atpdx -I320 protein \cite{Hanes2008b}. Combined with the structures of the wild type protein at different stages in the catalytic cycle it becomes clear that the enzyme uses the formation of the I320 intermediate as a mechanism for transferring the reaction from the P1 to the P2 site. The incorporation of ammonia appears to be coupled to a change in the position of of Lys 166, required for I320 formation. This coupling may be achieved by ammonia disrupting the position of Met 162 which is located on the same $\beta$ strand as as Lys 166 and forms part of the transient, methionine lined, ammonia tunnel as first proposed by Guedez \textit{et al} \cite{Guedez2012}.    

\begin{figure}[!htbp]
\centering
\begin{subfigure}{.225\textwidth}
  \centering
  \includegraphics[width=3.5cm, keepaspectratio]{/Users/matt/Dropbox/plps/figures/Fig1C/R5P/sw5-4cartoon8strand150817}
  \caption{}
\end{subfigure}
\begin{subfigure}{.225\textwidth}
  \centering
  \includegraphics[width=3.5cm, keepaspectratio]{/Users/matt/Dropbox/plps/figures/Fig1C/320/ma4-1cartoon8strand150817}
  \caption{\label{fig:I320_panel}}
\end{subfigure}
\begin{subfigure}{.225\textwidth}
  \centering
  \includegraphics[width=3.5cm, keepaspectratio]{/Users/matt/Dropbox/plps/figures/Fig1C/G3P/vw47-2_cartoonpanel}
  \caption{\label{fig:G3P_panel}}
\end{subfigure}
\begin{subfigure}{.225\textwidth}
  \centering
  \includegraphics[width=3.5cm, keepaspectratio]{/Users/matt/Dropbox/plps/figures/Fig1C/PLP/ma17-3cartoon8strand150817}
  \caption{\label{fig:PLP_panel}}
\end{subfigure}
\caption[Wild Type \atpdx~ structures in sequence]{(a) The wild type \atpdx -R5P structure (b) The wild type \atpdx -I320 structure (c) The wild type \atpdx -I320/G3P structure (d) The wild type \atpdx -PLP structure.\label{fig:cartoon_panel}}
\end{figure}  


%The elimination of the R5P phosphate group during I320 formation creates space for G3P to bind in the P1 site which leads to release of the carbohydrate from Lys 98 and transfer of the reaction to the P2 site. It is unclear at what stage in the reaction after binding of G3P the reaction is trasferred to the P2 site and whether closue of the pyridine ring occurs in the P1 site or the P2 site.    

%Enzymes displaying substrate ambiguity can bind different molecules in the same binding site and are not rare, however, they typically catalyse similar reactions on those substrates. The ability of enzymes to display substrate ambiguity coupled with gene duplications has been identified as an important means for organisms to evolve new metabolic pathways. In the case of the P1 site the chemistry performed on R5P and G3P is dissimilar, and the binding of R5P followed by I320 formation appears to be a pre-requisite for binding of G3P.  

%On the basis of the crystal structures it has been possible to produce an updated chemical mechanism for biosynthesis of PLP by Pdx1. This mechanism is based on the mechanism produced by Hanes \textit{et al} with changes made to the structure of the I320 and I320/G3P intermediate states (Figure \ref{fig:R5P_isomers}, \ref{fig:R5PtoI320}, \ref{fig:I320toG3P}, \ref{fig:G3PtoPLP}) \cite{Hanes2008b}. At several steps in the mechanism catalytic residues are assigned as participating in acid-base catalysis, donating and withdrawing protons from the intermediates. While the side chain shown to be participating in catalysis has been shown as that of an acidic aspartate or glutamate residue the imidazole ring of histidine side chains may also be involved.

%Chapter \ref{ch:Mutant_Results} details the results of the experiments performed to investigate the role of specific residues in the catalytic mechanism of Pdx1 and to trap the enzyme in intermediate states that are not possible to trap with the wild type protein.   
\clearpage
\section{Wild type \atpdx ~crystallographic statistics}
\begin{table}[!ht]
  \centering
\begin{tabular}{|p{4cm}||P{3.75cm}|P{3.75cm}|}
 \hline
 \multicolumn{3}{|c|}{Crystallographic Statistics for Wild Type \atpdx ~Structures} \\
 \hline
 \multicolumn{1}{|l|}{Protein Name (Dataset)} &\textit{At}Pdx1.3WT-R5P (sw5-4)&\textit{At}Pdx1.3WT-I320 (ma4-1)\\
 \hline
 Data Collection   &ID14-1 (ESRF) ~~~~~~~~~~~~23/06/2008&I04-1 (DLS) ~~~~~~~~~~~30/01/2012\\
 Space group &R3&R3\\
 Unit cell &176.7,176.7,114.6&178.5,178.5,116.3\\
 Resolution    &24.78-1.91 (1.94-1.91)&52.2-1.73 (1.76-1.73)\\
 R\textsubscript{merge}&5.8 (41.6)&5.1 (39.1)\\
 CC\sfrac{1}{2}&0.996 (0.812)&0.997 (0.841)\\
 \sfrac{I}{$\sigma$(I)}&12.2 (2.7)&9.3 (1.9)\\
 Completeness (\%)   &99.9 (100.0)&98.1 (99.7)\\
 Multiplicity    &3.9 (3.9)&2.9 (2.9)\\
 Unique Reflections    &106517 (5302)&148847 (7132)\\%(5302) (7132)
 Wilson B-factor    &16.3&16.7\\
 R\textsubscript{work}/R\textsubscript{free}&0.158/0.189&0.166/0.193\\ 
 \hline
 \textbf{Number of atoms} & &\\
 Protein    &8405&8548\\
 Ligand    &124&68\\
 Water    &859&1043\\
 \hline
 \textbf{Ramachandran} & &\\
 Preferred &1100 (98.8\%) &1092 (99.5\%)\\ 
 Allowed &13 (1.2\%) &6 (0.5\%)\\ 
 Outliers &0 (0.0\%) &0 (0.0\%)\\ 
 \hline 
 \textbf{B-factors} & &\\
 Protein &29.4&22.8\\
 Ligand &23.8&30.4\\
 Water &32.5&33.9\\
 \hline
 \textbf{RMS Deviations} & &\\
 Bond Lengths (\si{\angstrom}) &0.006&0.006\\
 Bond Angles (\si{\degree}) &0.800&0.805\\
 \hline
\end{tabular}
  \caption[Crystallographic statistics for wild type \atpdx -R5P and I320 structures]{Table of crystallographic statistics for wild type \atpdx -R5P and \atpdx -I320 structures}
\end{table}

\newpage
\begin{table}[!ht]
  \centering
\begin{tabular}{|p{4cm}||P{4.25cm}|P{4.25cm}|}
 \hline
 \multicolumn{3}{|c|}{Crystallographic Statistics for Wild Type \atpdx ~Structures} \\
 \hline
 \multicolumn{1}{|l|}{Protein Name (Dataset)} &\textit{At}Pdx1.3WT-I320/G3P (vw47-2)&\textit{At}Pdx1.3WT-PLP (ma17-3)\\
 \hline
 Data Collection   &ID23-1 (ESRF) ~~~~~~~~~~ 05/09/2009 &ID14-1 (ESRF) ~~~~~~~~~~~~ 04/12/2011\\
 Space group &R3&R3\\
 Unit cell &178.3,178.3,114.8&178.6,178.6,116.9\\
 Resolution    &44.57-1.90 (1.93-1.90)&50-1.61 (1.64-1.61)\\
 R\textsubscript{merge}&11.5 (69.9)&4.7 (46.6)\\
 CC\sfrac{1}{2}&0.977 (0.447)&0.998 (0.799)\\
 \sfrac{I}{$\sigma$(I)}&6.1 (1.3)&9.0 (1.5)\\
 Completeness (\%)   &99.4 (97.8)&100.0 (100.0)\\
 Multiplicity    &3.2 (3.0)&2.8 (2.8)\\
 Unique Reflections    &107077 (5240)&172020 (8994)\\
 Wilson B-factor    &26.9&15.7\\
 R\textsubscript{work}/R\textsubscript{free} &0.184/0.226&0.148/0.170\\ 
 \hline
 \textbf{Number of atoms} & &\\
 Protein    &8164&8495\\
 Ligand    &156&144\\
 Water    &508&1123\\
 \hline
 \textbf{Ramachandran} & &\\
 Preferred &1019 (98.5\%)&896 (98.9\%)\\ 
 Allowed &13 (1.3\%)&10 (1.1\%)\\ 
 Outliers &3 (0.3\%) &0 (0.0\%)\\ 
 \hline 
 \textbf{B-factors} & &\\
 Protein &32.6&25.6\\
 Ligand &40.4&24.4\\
 Water &36.6&37.3\\
 \hline
 \textbf{RMS Deviations} & &\\
 Bond Lengths (\si{\angstrom}) &0.027&0.006\\
 Bond Angles (\si{\degree}) &2.33&0.796\\
 \hline
\end{tabular}
  \caption[Crystallographic statistics for wild type \atpdx ~structures]{Table of crystallographic statistics for wild type \atpdx -I320/G3P and \atpdx -PLP structures.}
\end{table}   





     